\subsection*{Name}
lwcat - postprocess output of plt to make PostScript, EPS, PDF or PNG

\subsection*{Synopsis}
\texttt{plt -T lw} ... \texttt{$|$ lwcat} [ \textit{options ...} ] 
\subsection*{Description}


\texttt{lwcat} collects the PostScript
output of \textsf{\texttt{plt}(1)} and adds a prolog and epilog to create a complete PostScript
document (or another format, if appropriate options have been selected).
 It is possible to concatenate the outputs of two or more \texttt{plt} runs to be
processed as a single job by \texttt{lwcat};  see the \textit{plt Tutorial and Cookbook}
for details. 
\subsubsection*{Output format}

By default, \texttt{lwcat} sends its output directly to
the default printer via \texttt{lpr}.  These options may be used to modify this behavior:
\begin{description}
\item [\texttt{-no} ] Send the output to the printer, but don't eject the page (use this option
if you wish to overlay the output with additional material to be produced
by another program). 
\item [\texttt{-ps} ] Write PostScript to the standard output (not to
the printer). 
\item [\texttt{-psv} ] Write PostScript to a temporary file and view it with
\texttt{gv} (ghostscript). 
\item [\texttt{-gv} ] Same as \texttt{-psv}. 
\item [\texttt{-eps} ] Write EPSF (encapsulated PostScript
format) to the standard output.  Note that this is only a close approximation
to EPSF;  it is acceptable to LaTeX's epsfig package, at least. 
\item [\texttt{-pdf} ] Write
PDF (portable document format) to the standard output. 
\item [\texttt{-png} ] Write PNG (portable
network graphics) format to the standard output.  
\end{description}

\subsubsection*{Window options}

By default,
the output appears within a 6.75x6 inch (171x152 mm) window, the lower left
corner of which is positioned 1 inch (25.4 mm) from the left edge and 3.5
inches (89 mm) from the bottom edge of the page.  The following options
may be used to modify the size, location, and orientation of the output:
\begin{description}
\item [\texttt{-landscape} ] Use landscape mode (rotate plot 90 degrees counterclockwise).

\item [\texttt{-sq} ] Plot in a 6x6 inch (152x152 mm) square window, 1.25 inches (32 mm) from
the left edge and 3.5 inches (89 mm) from the bottom edge of the page. 
\item [\texttt{-t}
] Plot in a 6.25x6.25 inch (159x159 mm) square window, positioned as for \texttt{-sq}.

\item [\texttt{-t2} ] Plot in a 6.25x4 inch (159x102 mm) window, positioned as for \texttt{-sq}. 
\item [\texttt{-CinC}
] Plot in a 4.75x3.15 inch (121x80 mm) window, positioned as for \texttt{-sq}. 
\item [\texttt{-sq2} ] Plot
in a 4.5x5.5 inch (114x140 mm) window, 2.5 inches (63 mm) from the left and
bottom edges of the page. 
\item [\texttt{-v} ] Plot in a 7x9.5 inch (178x241 mm) window, 0.75
inches (19 mm) from the left and bottom edges of the page (centered on
a US letter sheet). 
\item [\texttt{-v2} ] Plot in a 7x8.5 inch (178x216 mm) window, positioned
as for \texttt{-v}. 
\item [\texttt{-va4} ] Plot in a 190x275 mm window, centered on an A4 sheet. 
\item [\texttt{-full}
] Plot in a 7.5x10 inch (191x254 mm) window, centered on a US letter sheet.

\item [\texttt{-slide} ] Plot in a 7.5x5 inch (191x127 mm) window, 0.5 inches (12.7 mm) from
the left edge and 3 inches (76 mm) from the bottom edge of the page (3:2
aspect ratio, as for 35 mm slides). 
\item [\texttt{-screen} ] Plot in a 7.5x5.625 inch (191x143
mm) window, 0.5 inches (12.7 mm) from the left edge and 2.375 inches (60 mm)
from the bottom edge of the page (4:3 screen aspect ratio). 
\item [\texttt{-golden} ] Plot
in a 7.5x4.635 inch (191x118 mm) window, 0.5 inches (12.7 mm) from the left
edge and 3.365 inches (85 mm) from the bottom edge of the page (the aspect
ratio is approximately the "golden ratio", (1+sqrt(5))/2 = 1.61803  
\end{description}


Other
window options can be easily added;  see the source for \texttt{lwcat} for details.
 
\subsubsection*{Copies}

By default, \texttt{lwcat} prints a single copy.  Multiple copies can be produced
using the options \texttt{-c2}, \texttt{-c3}, \texttt{-c4}, \texttt{-c5}, and \texttt{-c6};  when using a PostScript printer,
this will almost always be much faster than rerunning \texttt{lwcat}, since the
document is downloaded and rasterized only once when using these options.
 To print more than 6 copies, repeat or combine these options as needed.

\subsection*{Files}
\begin{description}
\item [\texttt{/usr/lib/ps/plt.pro} ] PostScript prolog 
\end{description}

\subsection*{See Also}


\textsf{\texttt{plt}(1)} 
\subsection*{Availability}
\texttt{lwcat}
is available as part of the \texttt{plt} package in PhysioToolkit (see \texttt{SOURCE} below)
under the GPL. 
\subsection*{Authors}
Paul Albrecht and George B. Moody (\texttt{george@mit.edu}) 
\subsection*{Source}
\texttt{http://www.physionet.org/physiotools/plt/src/lwcat}

\
