\subsection*{Name}
imageplt - plot a greyscale image 
\subsection*{Synopsis}
\texttt{imageplt -d} \textit{nrows} \textit{ncols} [ \textit{options
...} ] [ \textit{file} ] 
\subsection*{Description}


\texttt{imageplt} provides a simple way to plot a greyscale
image using \textsf{\texttt{plt}(1)}.  The required arguments, \textit{nrows} and \textit{ncols}, specify the
numbers of rows and columns in the image.  The input \textit{file} (or the standard
input, if no input file is specified) contains only the grey levels for
each pixel (0 = white, 1 = black).  Each entry is an ASCII-coded decimal
floating point number, separated from adjacent entries by whitespace (one
or more spaces, tabs, or newlines).  The first \textit{nrows} entries are the grey
levels for column 0 of the image, botttom to top, and each successive column
from left to right of the image follows.  If \textit{nrows} is small, it may be convenient
to arrange the image file in columns and rows corresponding to those of
the image, but this is not necessary.  In no case should the length of a
line of input exceed 50000 bytes (defined as MAXLEN in the source). 

\textit{Options}
include: \begin{description}
\item [\texttt{-n} ] Generate a negative image (1 = white, 0 = black). 
\item [\texttt{-x} \textit{xmin} \textit{xmax}
] Specify the range of the x-coordinates (default: \textit{xmin}=0, \textit{xmax}=\textit{nrows}-1). 
\item [\texttt{-y}
\textit{ymin} \textit{ymax} ] Specify the range of the y-coordinates (default: \textit{ymin}=0, \textit{ymax}=\textit{ncols}-1).

\end{description}


The output of \textit{imageplt} is text in three columns, to be plotted using the
\texttt{-pc} option of \texttt{plt}, as in: \begin{description}
\item [imageplt -d 10 10 foo $|$ plt 0 1 2 -pc ] 
\end{description}

\subsection*{See Also}


\textsf{\texttt{plt}(1)},
\textsf{\texttt{pltf}(1)} 
\subsection*{Availability}
\texttt{imageplt} is available as part of the \texttt{plt} package in
PhysioToolkit (see \texttt{SOURCES} below) under the GPL. 
\subsection*{Author}
George B. Moody (\texttt{george@mit.edu})

\subsection*{Source}
\texttt{http://www.physionet.org/physiotools/plt/plt/misc/imageplt.c} 
\
